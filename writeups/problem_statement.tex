\documentclass[12pt]{article}
\usepackage[inline]{asymptote}

\title{recording msprime `coalescence records' from forward time simulations}
\author{Jaime Ashander}
\date{}
\begin{document}
\maketitle

\begin{asydef}
//
// Global Asymptote definitions
//
usepackage("bm");
texpreamble("\def\V#1{\bm{#1}}");
\end{asydef}

{\bf Goal} Our goal is to encode the ARG produced  from a simulation that
proceeds forward in time in {\tt msprime}-style coalesence records.
One way to do this is to simulate forward, sample, and then work backwards.
In theory, one could also produce "partial'' records going forward.
Up to present, this is what we have done --- producing records as described in
{\it Going forward} below and augmenting any singleton remining with a
``phantom'' sibling (all having the same ID.
These can be loaded in to {\tt msprime} but don't actually meet the specs.
It seems that to meet {\tt msprime} format requirements we will still need to
traverse backward from a given set of samples to prune off incomplete records.

{ \bf A scenario}
Below each chromosome is labeled on the left with an ID; a single breakpoint,
labelled below the chromosome, represents a recombination during the meiosis
that produced the chromosome.
Above, the haplotypes are labelled by the ID for the parent that produced them.

Consider a lineage descended from three parent chromosomes: $1,2,3$.
Mating between these produced a first generation of two half-sib individuals
$(4,6)$ and $(5,7)$.
A subsequent mating between these produced gametes $8$ and $9$ in the second generation.

\begin{center}
	\begin{asy}[width=\the\linewidth,inline=true]
		// positioning info
		real g0 = 0;
		real g1 = -0.2;
		real g1b = -0.325;
		real g2 = -.6;
		real col0 = 0.5;
		real col1 = 1.5;
		real chromlen = 0.5;
		pair l = (chromlen, 0);
//		pair final = (0, -.5);

		//chromosome labels
		pair[] n;
		n[1] = (0, g0);
		n[2] = (1, g0);
		n[3] = (2, g0);
		// gen1
		n[4] = (col0, g1);
		n[5] = (col1, g1);
		n[6] = (col0, g1b);
		n[7] = (col1, g1b);
		// gen2
		n[8] = (col0, g2);
		n[9] = (col1, g2);

		//breakpoints
		pair bpmark = (0, 0.025); // formatting mark
		pair[] bp;
		bp[4] = (0.5 * chromlen, 0);
		bp[5] = (0.25 * chromlen, 0);
		bp[6] = (0.0 * chromlen, 0);
		bp[7] = (0.0 * chromlen, 0);
		bp[8] = (0.75 * chromlen, 0);
		bp[9] = (0.3 * chromlen, 0);

		//parents
		pair rightfmt = (.05, .025);
		pair leftfmt = (-.05, .025);
		pair[] parents;
		parents[4] = (1, 2);
		parents[5] = (3, 2);
		parents[6] = (0, 2);
		parents[7] = (0, 3);
		parents[8] = (5, 6);
		parents[9] = (7, 6);

		for (int i = 1; i <= 3; ++i) {
			dot(format("$%d$", i), n[i], W);
			draw(n[i]--l + n[i]);
		}
		pair parent;
		pair bploc;
		for (int i = 4; i <= 9; ++i) {
			dot(format("$%d$", i), n[i], W);
			draw(n[i]--l + n[i]);
			bploc = bp[i] + n[i];
			draw(bploc--bploc + bpmark);
			label(format("$x_%d$", i), bploc, S);
			parent = parents[i];
			label(format("$%f$", parent.y), bploc + bpmark + rightfmt);
			if (parent.x != 0.0) {
				label(format("$%f$", parent.x), bploc + bpmark + leftfmt);
			}

		}
//		pair endline = (0, g2)  + (0, -0.25);
//		draw(endline--endline +(3, 0));
//		int idx;
//		for (int i=8; i <=9; ++i) {
//			dot(format("$%d$", i), n[i] + final, W);
//			draw(n[i] + final--final + l + n[i]);
//			//self
//			bploc = bp[i] + n[i] + final;
//			draw(bploc--bploc + bpmark);
//			label(format("$x_%d$", i), bploc, N);
//
//			parent = parents[i];
//			//l parent
//			idx = (int) parent.x;
//			bploc = bploc = bp[idx] + n[i] + final;
//			draw(bploc--bploc + bpmark);
//			label(format("$x_%d$", (int) parent.x), bploc, S);
//
//			//r parent
//			idx = (int) parent.y;
//			bploc = bploc = bp[idx] + n[i] + final;
//			draw(bploc--bploc + bpmark);
//			label(format("$x_%d$", (int) parent.y), bploc, N);
//		}
	\end{asy}
\end{center}

% All the accumulated recombination breakpoints are labelled on these final gametes, below the large horizonal line.

We describe coalescence records in the format $(left, right, node, (children), time)$ of {\tt
	msprime}.
Label the right endpoint of the chromosome by $L$ and the left by
$0.0$. Further, label the current generation $0$, the parent $-1$ and the
grandparent $-2$

\subsection*{Going forward}

We could imagine constructing the records going forward in time.
Each meiosis would produce some information, not all of it useful.

From the first generation, there is one singleton coalescence to $1$,
\begin{verbatim}
	A = (0, x_4, 1, (4, ), -2)
\end{verbatim}
a couple different coalescences to $2$ and a singleton left over
\begin{verbatim}
	B = (0, x_5, 2, (6, ), -2)
	C = (x_5, x_4,  2, (5, 6), -2)
	D = (x_4, L,  2, (4, 5, 6), -2)
\end{verbatim}
and one coalesce to $3$ with one singleton left
\begin{verbatim}
	E = (0, x_5, 3, (5, 7), -2)
	F = (x_5, L, 3, (7, ), -2)
\end{verbatim}

From the second generation, there is one coalesence to $6$ and some singletons
\begin{verbatim}
	G = (x_8, L,  6, (8, 9), -1)
	H = (0, x_8, 5, (8, ), -1)
	I = (0, x_9, 7, (9, ), -1)
	J = (x_9, x_8,  6, (9, ), -1)
\end{verbatim}

\subsubsection*{Combining}

Note that while going forward  we cannot discard records that will not be used
as we have no {\it a priori}  criterion to do so.  For example, records relating
to chromosome $4$ are never used in sampling coalescence from $8$ and $9$ but
this is not known going forward.

\subsection*{Going backward}
Moving back in time, in the parent generation, a haplotype on the right coalesces
in chromosome $6$,
\begin{verbatim}
	A' = (x_8, L, 6, (8, 9), -1)
\end{verbatim}
The rest of the chromosome has yet to coalesce,
\begin{verbatim}
	B' = (0, x_8, 5, (8, ), -1)
	C' = (0, x_9, 7, (9, ), -1)
	D' = (x_9, x_8, 6, (9, ), -1)
\end{verbatim}

In the grandparent generation, the other parts coalesce
(Assume the recombination points in chromosome $5$ and $9$ coincide and let $a =
x_5 =x_9$.)
\begin{verbatim}
	F' = (0, a, 3, (5, 7), -2)
	G' = (a, x_8, 2, (5, 6), -2)
\end{verbatim}

\subsubsection*{Combining}

The singleton records {\tt B' \dots G'} combine to form two complete
records (noting that {\tt B'} encodes two records split at {\tt a}), 
\begin{verbatim}
	(0, a, 3, (8, 9), -2)
	(a, x_8, 2, (8, 9), -2)
\end{verbatim}
that together with {\tt A'} form a complete set of records. Renumbered in
ascending order from the leaves, these are
\begin{verbatim}
	(x_8, L, 2, (0, 1), -1)
	(0, a, 3, (0, 1), -2)
	(a, x_8, 4, (0, 1), -2)
\end{verbatim}
and they encode the following sparse trees (moving left to right across the
chromosome and encoding trees as integer vectors with {\tt -1} signifying the
root)
\begin{verbatim}
	0, a: (3, 3, NA, -1, NA)
	a, x_8: (4, 4, NA, NA, -1)
	x_8, L: (2, 2, -1, NA, NA)
\end{verbatim}

\pagebreak


\end{document}
